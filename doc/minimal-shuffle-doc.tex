\documentclass[11pt]{report} % Aptara syntax


%\usepackage{authblk}
\usepackage{amsmath}
\usepackage{amssymb}

\usepackage{graphicx}
\usepackage{subfigure}
\usepackage{comment}
\usepackage{listings}
\lstset{columns=flexible,basicstyle=\footnotesize}
\usepackage{lscape}
\usepackage{moreverb}
\usepackage[width=\textwidth]{caption}

\usepackage[colorlinks,bookmarksopen,bookmarksnumbered,citecolor=red,urlcolor=red]{hyperref}

\usepackage{geometry}

\usepackage{longtable}

\newcommand\BibTeX{{\rmfamily B\kern-.05em \textsc{i\kern-.025em b}\kern-.08em
T\kern-.1667em\lower.7ex\hbox{E}\kern-.125emX}}

\usepackage{booktabs,mathptmx,siunitx}

\newcommand{\picoql}{{\sc p}i{\sc co~ql}}
\newcommand{\Picoql}{Pi{\sc co~ql}}
\newcommand{\etal}{{\em et al.}}
%\newcommand*{\PHD}{}
%\newcommand*{\JOURNAL}{\PHD}
%\newcommand*{\REF}{}


% Document starts
\begin{document}


% Title portion
\title{Documentation of
an efficient shuffling scheme for the construction of mix-nets}
\author{Marios Fragkoulis}

\maketitle

\begin{abstract}

This is a documentation of the work carried out by Bayer and Groth~\cite{BayerG12}.
The authors describe a minimal argument for verifying the correctness of a shuffle 
operation assuming zero knowledge.
They evaluate their approach by verifying the correctness of a shuffle of
100000 ElGamal cipher texts.

The documentation starts by presenting important aspects of the argument.
The high-level workflow of the shuffle argument follows.
The third chapter is devoted to the multi-exponentiation argument
and the fourth to the product argument.
These are the constituents of the shuffle argument.
Finally, the appendix includes cryptographic definitions.

This documentation aims to be self-contained.
Thus, it includes verbatim text taken from the original paper.

\end{abstract}



\tableofcontents


\chapter{Aspects of the shuffle argument}

The purpose of this chapter is to introduce important aspects of the shuffle argument.
The notation that will be used throughout this report, which is identical to the notation used
in the paper is presented first.
Then the setup algorithm and security guarantees are described.
Section~\ref{sec:assm} documents in a formal manner (taken from the original paper) the 
assumptions the authors use to describe the strength of the shuffle argument.
This is the most important section of this chapter.
A description of the commitment scheme used in the shuffle argument follows.
Finally, a heuristic and a lemma used in the paper are presented.

\section{Notation}

%TODO: remove "we" throughout.
We write $y = A(x; r)$ when the algorithm $A$ on input $x$ and randomness $r$, outputs $y$. 
We write $y \leftarrow A(x)$ for the process of picking randomness $r$ at random and 
setting $y = A(x; r)$. 
We also write $y \leftarrow S$ for sampling $y$ uniformly at random from a set $S$.

We say a function $f :N \rightarrow [0,1]$ is negligible if $f(\lambda)=O(\lambda - c)$ for every 
constant $c>0$. We say $1-f$ is overwhelming if f is negligible. 
We will give a security parameter $\lambda$ written in unary as input to all parties in our protocols. 
Intuitively, the higher the security parameter the more secure the protocol. 
Formally, we define security in the following sections by saying an adversarial algorithm 
has negligible (in the security parameter) probability of succeeding in its attack.

For vectors of group elements, we write $\vec{c} \vec{y} = (x_1 y_1,..., x_n y_n)$ for the 
entry-wise product and correspondingly $x^z = (x^z_1,..., x^z_n)$. 
We write $\vec{x}_{\pi}$ if the entries of vector $\vec{x}$ are permuted by the 
permutation $\pi$, i.e., $\vec{x}_{\pi} =(x_{\pi(1)},...,x_{\pi(n)})$.
For vectors of field elements, we use the standard inner product 
$\vec{x} \vec{y} = \Pi^n_{i=1} x_i y_i$.

\section{Setup algorithm and security guarantees}

The argument consists of a prover P and a verifier V, both of which are probabilistic 
polynomial time interactive algorithms.
We assume the existence of a probabilistic polynomial time setup algorithm G 
that when given a security parameter $\lambda$ returns a common reference string $\sigma$.

In our case, the common reference string will be $\sigma = (pk, ck)$, where pk and ck are public 
keys for the ElGamal encryption scheme and the generalized Pedersen commitment scheme 
described previously. 
The encryption scheme and the commitment scheme may use different underlying groups, 
but we require that they have the same prime order q. We will write G for the group used 
by the commitment scheme and write H for the ciphertext space.

The setup algorithm can also return some side-information that may be used by an adversary; 
however, we require that even with this side-information the commitment scheme should remain 
{\em computationally binding}.\footnote{ i.e., semantically secure; a definition is included 
in Appendix~\ref{app:defs}.}
The side-information models that the keys may be set up using some multi-party computation 
protocol that leaks some information, the adversary may see some decryptions or even learn 
the decryption key, etc. 
Our protocol for verifying the correctness of a shuffle is secure in the presence of such leaks 
as long as the commitment scheme is computationally binding.

Let R be a polynomial time decidable ternary relation, we call w a witness for a statement x 
if $(\sigma, x, w) \in R$. We define the languages
\begin{equation}
L_\sigma :={x | \exists w : (\sigma,x,w) \in R}
\end{equation}
as the set of statements x that have a witness w for the relation R.

The public transcript produced by P and V when interacting on inputs s and t is denoted by
\begin{equation}
tr \leftarrow \langle P (s), V (t) \rangle
\end{equation}
The last part of the transcript is either accept or reject from the verifier. 
We write 
\begin{equation}
\langle P (s), V (t) \rangle = b, b \in {0, 1}
\end{equation}
for rejection or acceptance.

\section{Assumptions and strength of the shuffle argument}
\label{sec:assm}

%TODONE(1): Explain the authors' workflow in demonstrating the shuffle
% argument's process and its strength.
The authors use four assumptions that define the setting in which the proposed
shuffle argument verifies the correctness of a shuffle.
Effectively the assumptions characterize the strength of the argument.
The assumptions are defined in order of reducing generality.


\subsection{Perfect completeness and computational soundness}

The triple (G, P, V ) is called an argument for a relation R with {\em perfect completeness}
if for all non-uniform polynomial time interactive adversaries A we have:
\begin{itemize}

\item Perfect completeness:
\begin{equation}
Pr[(\sigma,hist) \leftarrow G(1^\lambda);(x,w) \leftarrow A(\sigma ,hist) : (\sigma ,x,w) \notin R 
\text{ or } \langle P(\sigma ,x,w),V(\sigma,x)\rangle = 1] = 1 
 \end{equation}
The equation states that exactly one of two mutually exclusive events will happen with 
probability 1: either the triple $(\sigma, x, w)$ does not belong to the space of the relation R
or the verifier accepts the transcript.
Perfect completeness in fact means that the verifier will never reject a valid proof.

\item Computational soundness:
\begin{equation}
Pr[(\sigma ,hist) \leftarrow G(1^\lambda);x \leftarrow A(\sigma ,hist) : x \notin L_\sigma \text{ and } 
\langle A,V(\sigma,x)\rangle = 1] \approx 0
\end{equation}
The equation states that the probability of accepting the proof of a statement x 
that does not belong to $L_s$, i.e., a witness $w : (\sigma, x, w) \in R$ cannot
be extracted, approximates 0.

\end{itemize}

\subsection{Public coin argument}

An argument (G, P, V) is called {\em public coin} if the verifier chooses his messages uniformly at 
random and independently of the messages sent by the prover, i.e., the challenges correspond 
to the verifier's randomness $\rho$.

\subsection{Perfect special honest verifier zero knowledge argument}

A public coin argument (G, P, V) is called a {\em perfect special honest verifier zero knowledge 
({\sc shvzk}) argument} for R with common reference string generator G if there exists a 
probabilistic polynomial time simulator S such that for all non-uniform polynomial time 
adversaries A we have:
\begin{multline}
Pr[(\sigma, hist) \leftarrow G(1\lambda); (x, w, \rho) \leftarrow A(\sigma, hist); \\
tr \leftarrow \langle P(\sigma,x,w),V(\sigma,x;\rho)\rangle : (\sigma,x,w) \text{ in } R \text{ and } A(tr) = 1] = \\
Pr[(\sigma, hist) \leftarrow G(1\lambda); (x, w, \rho) \leftarrow A(\sigma, hist); \\ 
tr \leftarrow S(\sigma,x,\rho) : (\sigma,x,w) \text{ in } R \text{ and } A(tr) = 1]
\end{multline}

This means that if there exists a probabilistic polynomial time simulator $S$ that 
accepts a statement $x$ and extracts a witness $w$ for it with the same probability as 
the public coin argument $(G, P, V)$, then $(G, P, V)$ is {\sc shvzk}.

To construct a fully zero-knowledge argument secure against arbitrary verifiers in the 
common reference string model one can first construct a {\sc shvzk} argument and then 
convert it into a fully zero-knowledge argument~\cite{Groth04, GarayMY03}. 
This conversion has constant additive overhead, so it is very efficient and allows us to 
focus on the simpler problem of getting {\sc shvzk} against honest verifiers.

To define an argument of knowledge we follow the approach of Groth and Ishai~\cite{GrothI08} 
and do it through witness-extended emulation first introduced by Lindell~\cite{Lindell01}. 
This definition informally says that given an adversary that produces an acceptable argument 
with some probability, {\em there exists an emulator that produces a similar argument with the same 
probability and at the same time provides a witness w}.

\subsection{Witness extended emulation for argument}
\label{sub:witness}

A public coin argument (G, P, V) has {\em witness extended emulation} if for all deterministic 
polynomial time $P^*$ there exists an expected polynomial time emulator X such that for 
all non-uniform polynomial time adversaries A we have
\begin{multline}
Pr[(\sigma, hist) \leftarrow G(1\lambda); (x, s) \leftarrow A(\sigma, hist);
tr \leftarrow \langle P^*(\sigma, x, s), V(\sigma, x)\rangle : A(tr) = 1] \approx 
Pr[(\sigma, hist) \leftarrow G(1\lambda); \\
(x, s) \leftarrow A(\sigma, hist); (tr, w) \leftarrow X^{\langle P^*(\sigma,x,s),
V(\sigma,x)\rangle} (\sigma, x, \rho) : A(tr) = 1 \\
\text{ and if tr is accepting then } (\sigma, x, w) \in R]
\end{multline}
In the definition, s can be interpreted as the state of $P^*$, including the randomness. 
So whenever $P^*$ is able to make a convincing argument when in state s, the emulator 
can extract a witness at the same time giving us an argument of knowledge. 
\textbf{This definition automatically implies soundness.}

\section{Algorithmic complexity of the shuffle verification argument}

%TODO(2): Explain the complexity

%A zero knowledge argument for the correctness of a shuffle of ElGamal cipher texts
%is proposed.
The argument's total cost consists of a communication part and a computation part.
The argument incurs sublinear communication complexity.
When shuffling N ciphertexts, arranged in an $m \times n$ matrix, the argument transmits
$O(m+n)$\footnote{ denotes the matrix's diagonals, which are $m+n-1$ to be precise (see
Table~\ref{tab:shuf}).} 
group elements giving a minimal communication complexity of $O(\sqrt N)$
if $m = n$.
The prover's computational complexity is $O(Nlogm)$ exponentiations and $O(N)$
if we allow a logarithmic number of rounds.


\section{The commitment scheme}

The original work uses a commitment scheme to freeze values such as a chosen permutation.
The commitment scheme used is a generalization of the Pedersen commitment
scheme~\cite{Pedersen92}.
For a cyclic group $\mathbb{G}$ of large prime order q, the general Pedersen commitment scheme
selects random generators $G_1,...,G_n,H$ of the group $\mathbb{G}$ and sets the commitment scheme
$ck = (\mathbb{G},G_1,...,G_n,H)$. To commit to n elements $(a_1,...,a_n)$ $\in$ $\mathbb{Z}^n_q$
we pick randomness r $\in$ $\mathbb{Z}_q$ and compute 
\begin{equation}
com_{ck} = H^r \Pi ^n_{i=1} G^{a_i}_i
\end{equation}
We can also commit to less than n elements; this is done by setting the remaining entries to
0.
The scheme assumes that interested parties have verified that commitments belong to the 
group $\mathbb{G}$.

The commitment is computationally binding under the discrete logarithm assumption, that is,
a non-uniform probabilistic polynomial time adversary has negligible probability of finding
two different openings of the same commitment.
Note: This seems to satisfy the {\sc ddh} assumption.

The commitment scheme is perfectly hiding since the commitment is uniformly distributed
in $\mathbb{G}$ no matter what the messages are.
Note: this seems to satisfy perfect secrecy, that is, confidentiality against arbitrary adversaries.

A commitment consists of a single group element no matter how big n is.
This means that the commitment scheme is length reducing; we can commit to n elements
with a single small commitment.
\textbf{This property is crucial to get sublinear communication cost.}

The generalized Pedersen commitment scheme is homomorphic; for all $\vec{a},\vec{b}$
$\in$ $\mathbb{Z}^n_q$ and r,s $\in$ $\mathbb{Z}_q$ we have:
\begin{equation}
com_{ck}(\vec{a};r)com_{ck}(\vec{b};s) = H^r \Pi ^n_{i=1} G^{a_i}_i \cdot
H^r \Pi ^n_{i=1} G^{b_i}_i = H^{r+s} \Pi ^n_{i=1} G^{a_i+b_i}_i = com_{ck}(\vec{a}+\vec{b};r+s)
\end{equation}


\section{Other useful math devices}

The authors refer to the Fiat-Shamir heuristic and the Schwartz-Zippel lemma
in the original paper.

\subsection{The Fiat-Shamir heuristic}

In the Fiat-Shamir heuristic the prover computes the public-coin challenges with a 
cryptographic hash function instead of interacting with a verifier. 
This makes it possible for the prover to compute the entire argument without any 
interaction, i.e., it is a non-interactive argument.

Non-interactivity is desirable in many applications. Consider for instance an election, 
where the authorities want to convince independent verifiers that the tally is correct. 
Using non-interactive arguments the authorities only need to compute the argument once 
and then they can send the same non-interactive argument to all the verifiers. 
If the argument was interactive on the other hand, they would have to interact with each 
verifier, which would increase the complexity of the protocol.

The Fiat-Shamir heuristic is known to be secure in the random oracle model, where the 
cryptographic hash-function is modeled as a random oracle that returns a uniformly random 
answer to inputs it has not seen before. 
Those who are satisfied with a security proof in the random oracle model can therefore 
obtain a significant saving in interaction and in the prover�s computation. 
For the sake of generality, we will describe the interactive protocols though.

\subsection{The Schwartz-Zippel lemma}
\label{sub:szl}

For completeness we will state a variation of the Schwartz-Zippel lemma that we will use 
several times. 

\begin{description}

\item Lemma 1 (Schwartz-Zippel). Let p be a non-zero multivariate polynomial of degree d 
over $Z_q$, then the probability of $p(x_1,...,x_n) = 0$ for randomly chosen 
$x_1,...,x_n \leftarrow \mathbb{Z}^*_q$ is at most $\frac{d}{q-1}$.

Given two multi-variate polynomials $p_1$ and $p_2$ we can test whether 
\begin{equation}
p_1(x1,..., xn) - p_2(x1,..., x_n) = 0 \text{ for random } x_1,...,x_n \leftarrow \mathbb{Z}^*_q
\end{equation}
 or not. 
This equation will always hold if p1 = p2, whereas if $ p1 \neq p2$ the
probability that the test pass is only $\frac{max(d1,d2)}{q - 1}$.

\end{description}


\chapter{High-level workflow of the shuffle argument}

Initially, the setting of the application is described.
Then the shuffle argument and the complete high-level workflow
are presented.
The chapter concludes with the strength of the shuffle argument.

\section{Setting}

Initially, voters commit their votes to the e-voting system in an encrypted form using the
ElGamal homomorphic encryption scheme.
To ensure anonymity the votes are permuted and re-encrypted.
%to unlink the relationship between voters
%and encrypted votes?.
So, the votes are input to the mix-net, which may consist of a number of mix-net servers 
that take turns in shuffling the messages.
If the encryption scheme is semantically secure, the output does not reveal the permutation 
of the messages.
Furthermore, because the votes are encrypted it is not possible to verify directly the 
correctness of a shuffle operation.
This could be a window of opportunity for malicious action.
A malicious mix-net server could substitute encrypted votes with encrypted votes for candidate
X.
To verify the correctness of a shuffle operation, the authors
use a zero-knowledge interactive argument.

\section{Shuffle argument}

%Consider the list of votes arranged in a square $n \times n$ matrix, then
We will give an argument of knowledge of a permutation $\pi \in \Sigma_N$
and randomness $\text{\{}\rho_i\text{\}}^N_{i=1}$ such that for given ciphertexts 
$\text{\{}C_i\text{\}}^N_{i=1}, \text{\{}C_{i'}\text{\}}^N_{i=1} \text{ we have } C_{i'} 
= C_{\pi(i)} E_{pk}(1; \rho_i)$.
%\footnote{the encryption function is 1; so, no re-encryption
%in the shuffling phase?} 
The shuffle argument combines: 

\begin{itemize}

\item a multi-exponentiation argument, which allows 
us to prove that the product of a set of ciphertexts raised to a set of committed 
exponents\footnote{they represent the chosen permutation} yields a 
particular ciphertext, and 

\item a product argument, which allows us to prove that a set of committed 
values, that is, the permutation used, has a particular product.

\end{itemize}

It is called zero-knowledge because it reveals nothing about the permutation of
the randomizers used.

\section{High-level workflow}

The workflow consists of two phases:

\begin{itemize}

\item pick a permutation and verify that this one is used across the shuffling 
operation by means of the product argument (see Section~\ref{sub:perm}) and,

\item verify the correctness of the shuffling operation itself by means of the
multi-exponentiation argument (see Section~\ref{sub:shuf}).

\end{itemize}

The authors use (the hierarchy of) assumptions defined in Section~\ref{sec:assm}
consistently throughout the paper to demonstrate the strength of the argument in 
terms of the surrounding setting e.g., honest vs arbitrary verifiers.
The authors sequentially demonstrate that the product argument, 
multi-exponentiation argument, and optimizations to the latter

\begin{itemize}

\item achieve perfect completeness,

\item are public coin {\sc shvzk} arguments, and

\item have witness-extended emulation.

\end{itemize}

Their objective is twofold:

\begin{itemize}

\item demonstrate that their argument verifies the correctness of a shuffle operation
assuming zero-knowledge, and

\item demonstrate that it achieves computational soundness.

\end{itemize}

For the first objective, the authors show first how their argument achieves 
perfect completeness.
Then they reduce the assumptions and escalate their argument to a public
coin {\sc shvzk} argument.
This can be turned into a zero-knowledge argument with additive overhead, hence
the first objective is accomplished.

For the second objective, the authors demonstrate extracting a witness for each
acceptable argument, which implies computational soundness 
according to the definition of witness extended emulation (see Section~\ref{sub:witness}).

\subsection{High-level workflow of the product argument}
\label{sub:perm}

The product argument is used to verify that a specific permutation is used
across the shuffling phase.
Table~\ref{tab:perm} presents the high-level workflow of the product argument.

\begin{longtable}[t]{|p{0.45\textwidth}p{0.45\textwidth}|}
%\caption{\mbox{Use case (UC) identifiers and descriptions and respective \picoql{} queries}}
\caption{ presents the high-level workflow of the product argument in terms of 
prover-verifier interaction.}
\label{tab:perm} 
\\ [-5pt]
    \hline
    \multicolumn{1}{|c}{{\bf The prover}} 
    & \multicolumn{1}{c|}{{\bf The verifier}}
    \\
    \hline
    -- wants to extract a witness $\pi \in \Sigma_N$, the permutation, and $\vec{\rho} \in \mathbb{Z}^N_q$
    such that $\vec{C}' = E_{pk}(\vec{1};\vec{\rho})\vec{C}_{\pi}$ given a public key $ _{pk}$
    and a commitment key $ _{ck}$ and statements $\vec{C}, \vec{C}' \in \mathbb{H}^N$ with
    $N=m \times n$.
    &
    \\
    &
    \\
    - picks $\vec{r} \leftarrow \mathbb{Z}^m_q$, sets $\vec{a} = \{\pi(i)\}^N_{i=1}$, 
    and computes $\vec{c}_A = com_{ck} (\vec{a}; \vec{r})$.
    &
    \\
    - sends an argument of knowledge of opening of the commitment, $\vec{c}_A$, to 
    permutation $1,...,N$ to the verifier, that is, it commits to $\pi(1),..., \pi(N)$.
    &
    \\
    &
    - sends a challenge $x \leftarrow \mathbb{Z}^*_q$ to the prover.
    \\
    - receives $x$, picks $s \in \mathbb{Z}^m_q$, sets $\vec{b} = \{x^{\pi(i)}\}^N_{i=1}$, computes  
    $\vec{c}_B = com_{ck}(\vec{b}; \vec{s})$.
    &
    \\
    - sends an argument of knowledge of opening of the commitment, $\vec{c}_B$, to 
    the permutation of $x^1,...,x^N$, that is, it commits to $x^{\pi(1)},..., x^{\pi(N)}$.
    &
    \\
    - {\em seeks to demonstrate that the two permutations are identical.}
    This would mean that the prover has a commitment to $x^1,..., x^N$
    permuted in an order that was fixed before the prover saw $x$.
    &
    \\
    &
    - sends random challenges $y, z \leftarrow \mathbb{Z}^*_q$ to check that the same permutation 
    has been used in both commitments.
    \\
    - defines $\vec{c}_{-z} = com_{ck} (-z,..., -z; \vec{0})$ and $\vec{c}_D=\vec{c}^y_A \vec{c}_B$
    and computes $\vec{d} = y\vec{a} + \vec{b}, \text{ } \vec{t} = y\vec{r} + \vec{s}$.
    &
    \\
    - computes commitments to $d_1-z = y\pi(1) + x^{\pi(1)} - z,...,d_N-z = y\pi(N) + x^{\pi(N)} - z$
    in a verifiable manner by using the homomorphic properties of the commitment
    scheme.
    &
    \\
    - sends $\Pi^N_{i=1}(d_i - z) \text{, } \vec{c_D}\vec{c_{-z}} = com_{ck}(\vec{d} - \vec{z}, \text{ } \vec{t})$. 
    &
    \\
    &
    - computes $\vec{c_A}^y\vec{c_B}\vec{c_{-z}} = 
    H^{\vec{r}}(\Pi^N_{i=1}G_i^{a_i})^y \text{ }
    H^{\vec{s}}\Pi^N_{i=1}G_i^{b_i}\vec{c_{-z}} =
    H^{y\vec{r}}\Pi^N_{i=1}G_i^{ya_i} \text{ } H^{\vec{s}}\Pi^N_{i=1}G_i^{b_i}\vec{c_{-z}} =
    H^{y\vec{r}+\vec{s}}\Pi^N_{i=1}G_i^{ya_i + b_i}\vec{c_{-z}} = 
    com_{ck}(y\vec{a} + \vec{b} - \vec{z}; y\vec{r} + \vec{s}) =
    com_{ck}(\vec{d} - \vec{z}; \vec{t}) = \vec{c_D}\vec{c_{-z}}$.
    \\
    &
    - computes $\Pi^N_{i=1}(yi + x^i - z)$.
    
    If they are the same as
    the received arguments of knowledge, it means that {\em the same permutation was used.}
    Effectively $\Pi^N_{i=1}(d_i-z) = \Pi^N_{i=1}(yi + x^i - z)$.
    \\
\hline
\end{longtable}

    Note that in $\Pi^N_{i=1}(d_i-z) = \Pi^N_{i=1}(y\pi(i) + x^{\pi(i)} - z)$, 
    $\pi(i)$ represents the permutation of the
    first commitment and $x^{\pi(i)}$ represents the permutation of the second commitment.
    Only if permutations match, i.e., $\pi(i)$ is the same within each term $y\pi(i) + x^{\pi(i)}$, will the identity
    $\Pi^N_{i=1}(y\pi(i) + x^{\pi(i)} - z) = \Pi^N_{i=1}(yi + x^i - z)$ hold.
    It will hold because the order of the $N$ terms has no effect in the product.
    This proves {\em perfect completeness} of the product argument. 

More formally, observe that we have two identical degree N polynomials in z since 
the only difference is that the roots have been permuted. 
The verifier does not know a priori that the two polynomials are identical but can by 
the {\em Schwartz-Zippel lemma} (Section~\ref{sub:szl}) deduce that the prover has 
negligible chance over the 
choice of $z$ of making a convincing argument unless indeed there is a permutation $\pi$ such 
that $d_1 = y\pi(1) + x^{\pi(1)},..., d_N = y\pi(N) + x^{\pi(N)}$. 
Furthermore, there is negligible probability over the choice of $y$ of this being true unless 
the first commitment contains $\pi(1),..., \pi(N)$ and the second commitment contains 
$x^{\pi(1)},..., x^{\pi(N)}$.


\subsection{High-level workflow of the multi-exponentiation argument}
\label{sub:shuf}

The multi-exponentiation argument is used to verify the correctness of the shuffling operation.
Continuing from Section~\ref{sub:perm}, the prover now has commitments to $x^{\pi(1)},..., x^{\pi(N)}$ 
and uses the multi-exponentiation 
argument to demonstrate that there exists a $\rho$ such that $\Pi^N_{i=1} C^{x^i}_i = 
E_{pk}(1; \rho) \Pi^N_{i=1} (C'_i)^{x^{\pi(i)}}$. 
Note that the verifier does not see the committed values and therefore does not 
learn what the permutation is. 
However, from the homomorphic properties of the encryption scheme the 
verifier can deduce $\Pi^N_{i=1} M^{x^i}_i = \Pi^N_{i=1} (M'_i)^{x^{\pi(i)}}$
for some permutation $\pi$ that was chosen before the challenge $x$ was sent to the prover. 
Taking discrete logarithms we have the polynomial identity 
$\Sigma^N_{i=1} log(M_i)x^i = \Sigma^N_{i=1} log(M'_{\pi^{-1}(i)})x^i$. 
There is negligible probability over the choice of x of this equality holding true unless 
$M'_1 = M_{\pi(1)},..., M'_N = M_{\pi(N)}$. 
This shows that we have a correct shuffle.

\begin{longtable}[t]{|p{0.45\textwidth}p{0.45\textwidth}|}
%\caption{\mbox{Use case (UC) identifiers and descriptions and respective \picoql{} queries}}
\caption{ presents the high-level workflow of the multi-exponentiation argument in terms of 
prover-verifier interaction.}
\label{tab:high-shuf} 
\\ [-5pt]
    \hline
    \multicolumn{1}{|c}{{\bf The prover}} 
    & \multicolumn{1}{c|}{{\bf The verifier}}
    \\
    \hline
    \multicolumn{2}{|c|}{(\textit{continuing from Table~\ref{tab:perm}})}
    \\
    - computes $\rho=-\vec{\rho} \vec{b}$ and then $E_{pk}(1; \rho)\vec{C}'^{\vec{b}}$ where
    $E_{pk}(1; \rho)\vec{C}'^{\vec{b}} =$
    
    $E_{pk}(\vec{1}; -\vec{\rho})^{\vec{b}} (E_{pk}(\vec{1}; \vec{\rho})\vec{C}_{\pi})^{\vec{b}} =$
    
    $\vec{C}^{\vec{b}}_{\pi} = 
    \Pi^N_{i=1} C^{x^{\pi(i)}}_{\pi(i)} =
    \vec{C}^{\vec{x}}$
    
    This proves perfect completeness of the multi-exponentiation argument.
    &
    \\
    &
    - sets $\vec{x} = (x, x^2,..., x^N)^T$ and computes $\vec{C}^{\vec{x}}$.

    \\
\hline
\end{longtable}


\subsection{Strength of the complete argument}

{\em Perfect completeness} follows from the perfect completeness of the product 
argument and the perfect completeness of the multi-exponentiation argument.

{\em Perfect {\sc shvzk}} follows from the fact that the commitments are perfectly hiding 
and the underlying arguments are perfect {\sc shvzk}.
To simulate the entire argument we can pick random commitments 
$\vec{c}_A, \vec{c}_B \leftarrow com_{ck}(0,...,0)$ which can be done without 
knowing the witness for the correctness of the shuffle. 
The simulator then runs perfect {\sc shvzk} simulations of the product and 
multi-exponentiation arguments.

Finally, we have to show that we have {\em witness-extended emulation}. 
In the witness-extended emulation we run the prover and verifier $\langle P^*, V\rangle$ 
with random challenges $x, y, z \in \mathbb{Z}^*_q$ and random challenges for the product and 
multi-exponentiation arguments. 
If we get an acceptable argument we have to extract a witness. 

We start by rewinding and running the witness-extended emulators for the product and the 
multi-exponentiation arguments. 
This gives us openings of $\vec{c}_B$ and $\vec{c}_D$. 
Since $\vec{c}_D = \vec{c}^y_A \vec{c}_B$ this allows us to compute openings
$\vec{a}, \vec{r}$ of $\vec{c}_A$.
We will now argue that with overwhelming probability the extracted openings of $\vec{c}_A$ 
must be of the form $\vec{a} = \{ \pi(i)\}^N_{i=1}$ for some permutation $\pi \in \Sigma_N$. 
Consider the situation after round 3 where the prover has sent $\vec{c}_B$. 
The witness-extended emulation of the multi-exponentiation argument gives us the 
opening $\vec{b}, \vec{s}$ of $\vec{c}_B$ showing that the opening of $\vec{c}_D$ must be of 
the form $\vec{d} = \vec{y}a + \vec{b}$.
 
The product argument shows that $\Pi^N_{i=1} (d_i - z) = \Pi^N_{i=1}(y_i + x_i - z)$. 
This has negligible probability over $z$ in succeeding unless there exists a permutation 
$\pi$ such that $d_i = \pi(i) + x^{\pi(i)}$. 
This shows $ya_i + b_i = y\pi(i) + x^{\pi(i)}$, which has negligible probability over $y$ of 
being true unless $a_i = \pi(i)$.
Furthermore, we then see that $b_i = x^{\pi(i)}$.
We now rewind and run the argument until we have extracted witnesses for the 
multi-exponentiation argument for $N$ random choices of $x$. 
Each choice $x_j$ gives us a witness containing $\rho^{(j)}$ such that

\begin{equation}
\vec{C}^{\vec{x}_j} = E_{pk} (1;\rho^{(j)}) \Pi^N_{i=1}(C'_i)^{x^{\pi(i)}_j} = 
E_{pk} (1;\rho^{(j)})\vec{C}'^{\vec{x}^j}_{\pi_{-1}}
\end{equation}
The $N \times N$ matrix
\begin{equation}
X = 
\begin{pmatrix}
x^1_1 & \cdots & x^1_N \\
\vdots &  & \vdots \\
x^N_1 & \cdots & x^N_N
\end{pmatrix}
\end{equation}

can be viewed as a submatrix of a transposed Vandermonde matrix. 
If $x_1,..., x_N$ are different then $X$ is invertible. 
We now have 
\begin{equation}
\vec{C} = (\vec{C}^X)^{X^{-1}} = (E_{pk}(\vec{1}; \vec{\rho}) \vec{C'}^X_{\pi^{-1}})^{X^{-1}} =
E_{pk}(\vec{1}; \vec{\rho}X^{-1}) \vec{C'}_{\pi^{-1}}
\end{equation}

This gives us a permutation $\pi$ and $\vec{\rho}' = (-\vec{\rho} X^{-1})\pi$ such that $\vec{C}' = E_{pk}(\vec{1}; \vec{\rho}')\vec{C}_{\pi}$.

This shows that the protocol is a public coin perfect {\sc shvzk} argument of knowledge of
$\pi \in \Sigma_N$ and $\vec{\rho} \in Z^N_q$.

The two arguments presented at high level in Sections~\ref{sub:perm} and~\ref{sub:shuf} 
can be run in parallel.
Furthermore, the multi-exponentiation argument can be started after the computation 
of the commitments $\vec{c}_B$.


\chapter{The multi-exponentiation argument}

%FIGURE

This chapter is devoted to the multi-exponentiation argument.
It documents its workflow in detail and presents the applied
optimizations.

\section{Detailed workflow}

Table~\ref{tab:shuf} presents the high-level workflow of the multi-exponentiation argument
in the form of prover-verifier interaction.

\begin{longtable}[t]{|p{0.45\textwidth}p{0.45\textwidth}|}
%\caption{\mbox{Use case (UC) identifiers and descriptions and respective \picoql{} queries}}
\caption{ presents the high-level workflow of the multi-exponentiation argument in terms of 
prover-verifier interaction.}
\label{tab:shuf} 
\\ [-5pt]
    \hline
    \multicolumn{1}{|c}{{\bf The prover}} 
    & \multicolumn{1}{c|}{{\bf The verifier}}
    \\
    \hline
    - computes $E_k = \Pi_{\substack{1 \leq i, j \leq m \\ j = (k - m) + 1}} \vec{C}^{\vec{a}_j}_i$
    given ciphertexts $\vec{C}_{11},..., \vec{C}_{mn}$ and $C$.
    The ciphertext $C = E_m$ is the product of the main diagonal and the other $E_k$'s 
    are the products of the other diagonals. 
    &
    \\
    - sends $E_1,..., E_{2m-1}$.

    uses a batch-proof to simultaneously convince the verifier that all the diagonal 
    products give their corresponding $E_k$:
    &
    \\
    &
    - selects a challenge $x \leftarrow \mathbb{Z}^*_q$ at random.
    \\
    - sets $\vec{x} = (x, x_2,..., x_m)^T$ and opens $\vec{c}^{\vec{x}}_A \text{ to }
    \vec{a} = \Sigma^m_{j=1} x^j\vec{a}_j$.
    &
    \\
    - computes $\Pi^m_{i=1} \vec{C}^{x^{m-i}\vec{a}}_i$.
    &
    \\
    &
    - computes 
    $\Pi^{2m -1}_{k = 1} E_k^{x^k}$ where
    $\Pi^{2m -1}_{k = 1} E_k^{x^k} = $

    $\Pi^{2m -1}_{k = 1} (\Pi^{m,m}_{\substack{i=1, j=0 \\ j=(k-m)+i}} \vec{C}^{\vec{a}_j}_i)^{x^k} = $
    
    $\Pi^{2m -1}_{k = 1} \Pi^{m,m}_{\substack{i=1, j=0 \\ k=j+m-i}} \vec{C}^{\vec{a}_j x^{k}}_i = $
    
    $\Pi^{m,m}_{\substack{i=1, j=0 \\ k=j+m-i}} \vec{C}^{\vec{a}_j x^{m-i+j}}_i = $
    
    $\Pi^{m,m}_{i=1, j=0} \vec{C}^{x^{m-i}x^j\vec{a}_j}_i = $
    
    $\Pi^{m}_{i=1} \vec{C}^{x^{m-i}\Sigma^m_{j=0} x^j\vec{a}_j}_i = $
    
    $\Pi^{m}_{i=1} \vec{C}^{x^{m-i}\vec{a}}_i$.
    
    This proves perfect completeness.
    \\
\hline
\end{longtable}

Since $x$ is chosen at random, the prover has negligible probability of convincing 
the verifier unless the $x^k$-related terms match on each side of the equality for all $k$. 
In particular, since $\vec{a} = \Pi^m_{j=1} \vec{x}^ja_j$ the $x^m$-related terms give us

\begin{equation}
C^{x^m} = \Pi^m_{i=1} \vec{C}^{x^{m-i} \Sigma_{\substack{1 \leq j \leq m \\ m=m-i+j}} x^j \vec{a}^j} =
\Pi^m_{i=1} \vec{C}^{x^{m-i} x^i \vec{a}_i}_i = (\Pi^m_{i=1} \vec{C}^{a^i}_i)^{x^m}
\end{equation}

and allow the verifier to conclude $C = \Pi^m_{i=1} \vec{C}^{\vec{a}_i}_i$.

Finally, to make the argument honest verifier zero-knowledge we have to avoid leaking 
information about the exponent vectors $\vec{a}_1,..., \vec{a}_m$. 
The prover therefore commits to a random vector $\vec{a}_0 \leftarrow Z^n_q$ and after 
she sees the challenge $x$ she reveals $\vec{a} = \vec{a}_0 + \Sigma^m_{j=1} x^j\vec{a}_j$. 
Since $\vec{a}_0$ is chosen at random this vector does not leak any information about the 
exponents.

\section{Optimizations}

A combination of Toom-Cook~\cite{Toom63} technique for multiplication of large integers 
and an interactive technique composes the most efficient scheme as the 
two techniques work well together.

\subsection{Toom-Cook technique for multiplication of very large integers}

The prover's computation involves the following product (originated from 
Table~\ref{tab:shuf}).

\begin{equation}
E_k = \Pi_{\substack{1 \leq i, j \leq m \\ j = (k - m) + 1}} \vec{C}^{\vec{a}_j}_i
\end{equation}

To compute this product $m^2n$ exponentiations are required.
This stems from the fact that the product
$\vec{C}^{\vec{a}_1}_1 \times \cdots \times \vec{C}^{\vec{a}_m}_m$
causes $m^2$ exponentiations and the product
$C^{a_{j1}}_{i1} \times \cdots \times C^{a_{in}}_{jn}$
causes another $n$ exponentiations, that is, $m^2n$ in total.
For large $m$ the cost is prohibitive.

Toom-Cook's technique for polynomial multiplication helps reduce this cost 
significantly.
The following form is an equivalent form of the same product.

\begin{equation}
\Pi^m_{i=1} \vec{C}^{x^{m-i}\vec{a}}_i = 
\Pi^{m}_{i=1} \vec{C}^{x^{m-i}\Sigma^m_{j=1} x^{j-1}\vec{a}_j}_i
\end{equation}

It is useful because it allows to model the product as two polynomials of
degree $m-1$.

\begin{equation}
\Pi^{m}_{i=1} \vec{C}^{x^{m-i}_i} \text{\hspace{2cm}} \Sigma^m_{j=1} x^{j-1}\vec{a}_j
\end{equation}

The idea then is to evaluate the product of the polynomials in $2m-1$\footnote{ We need
m points to solve an equation with m-1 variables. 
The polynomials introduce 
$(m-1) + (m-1)$ variables, hence we need  $(m-1) + (m-1) +1$ points.} points of 
our choice and find their coefficients using polynomial interpolation.
%TODO: give an example of polynomial interpolation
%            understand linear combinations
%            understand roots of unity

The expensive step in this computation is to compute 
\begin{equation}
\Pi^{m}_{i=1} \vec{C}^{\omega^{m-i}_0}, ... , \Pi^{m}_{i=1} \vec{C}^{\omega^{m-i}_{2m-2}}
\end{equation}

Inspired by the Toom-Cook method for integer multiplication,
we may for instance choose $\omega_0, \omega_1, ... , \omega_{2m-2}$ to be small integers. 
When m is small even the largest exponent $\omega_{2m-2}$ will remain small. 
For instance, if $m = 4$ we may choose $\omega_k \in \{0, -1, 1, -2, 2, -3, 3\}$, which
$k$ makes the largest exponent $\omega^{m-1}_k = 3^3 = 27$. 
This makes it cheap to compute each $\Pi^m_{i=1} \vec{C}^{\omega^{m-1}_k}_i$ because 
the exponents are very small.
The basic step of Toom-Cook sketched above can be optimized by choosing the evaluation 
points carefully. 
However, the performance degrades quickly as m grows. 
Using recursion it is possible to get subquadratic complexity also for large $m$, 
however, the cost still grows relatively fast. 
In the next section an interactive technique for reducing 
the prover�s computation is described. 

\subsection{Reduced computation, increased interaction}

\setcounter{MaxMatrixCols}{20}

Computing all $m^2$ entries of the matrix is useless.
The interest focuses on the product along the main diagonal
in order to show that $C = E_m = \Pi^m_{i=1} \vec{C}^{\vec{a}^i}_i$.

Let us explain the idea in the case of $m = 16$ as depicted in Matrix~\ref{eq:matrix}. 
We can divide the matrix into $4 \times 4$ blocks and only use the four blocks 
that are on the main diagonal. 
%Let us for now just focus on soundness and return to the question of honest verifier 
%zero-knowledge later. 
The prover starts by sending $E_0, E_1, E_2, E_3,$ 
$E_4, E_5, E_6$ that are the 
products along the diagonals of the elements
in the blocks that we are interested in. I.e., $E_0 = \Pi^4_{i=1} \vec{C}^{\vec{a}_{4i-3}}_{4i},...,
E_6 = \Pi^4_{i=1} \vec{C}^{\vec{a}_{4i}}_{4i-3}$ and $E_3 = C$ (see Matrix~\ref{eq:matrix}). 
The verifier sends a random challenge x and using the homomorphic properties of 
the encryption scheme and of
the commitment scheme both the prover and the verifier can compute
$\vec{C}'_1,...,\vec{C}'_4$ and $c_{A'1},...,c_{A'4}$ as 

\begin{equation}
\vec{C}' = \vec{C}^{x^3}_{4i-3}
\vec{C}^{x^2}_{4i-2} \vec{C}^x_{4i-1} \vec{C}_{4i}
 \text{ \hspace{2cm} }
c_{A'j} = c_{A_{4j-3}} c^x_{A_{4j-2}} c^{x^2}_{A_{4j-1}} c^{x^3}_{A_{4j}}
\end{equation}


%They can also both compute $C' = \Pi^6_{k=0} E^{x^k}_k$. 
%The prover and the verifier now engage in an {\sc shvzk}
%argument for the smaller statement 
%$C = \Pi^4_{i=1} \vec{C}'^{\vec{a}'_i}$. 
$C'$ represents the rows of interest in the blocks located on the
main diagonal while $c_{A'_j}$ represents the commitment respectively.
The prover can compute a witness for statement $C = \Pi^4_{i=1} \vec{C}'^{\vec{a}'_i}$ with
$\vec{a}'_i = \vec{a}_{4i-3} + x\vec{a}_{4i-2} + x^2\vec{a}_{4i-1} + 
x^3 \vec{a}_{4i}$; $a'_i$ represents the columns of interest in the blocks
located on the main diagonal. This shows


\begin{equation}
\label{eq:diag}
C^{x^3} \Pi^6_{i=1} E^{x^k}_{k=0, k \neq 3} = \Pi^4_{i=1}(\vec{C}^{x^3}_{4i-3}\vec{C}^{x^2}_{4i-2}\vec{C}^{x}_{4i-1}
\vec{C}_{4i})^{(\vec{a}_{4i-3} + \vec{x}a_{4i-2} + \vec{x}^2_{a4i-1} + x^3a_{4i})}.
\end{equation}


The verifier computes the product across the main diagonal while the prover 
computes the product across the blocks located on the main diagonal
row by row as shown in equation~\ref{eq:diag}.
A valid argument requires that these two products match.

Looking at the $x^3$-related terms, we see there is negligible chance of the equation holding for a 
random $x$ unless $\Pi^m_{i=1} \vec{C}^{\vec{a}^i}_i$ , which is what the prover 
wanted to demonstrate.

\begin{equation}
\label{eq:matrix}
\begin{pmatrix}
\mathbf{\vec{C}^{\vec{a}_1}_1} & \vec{C}^{\vec{a}_2}_1 & \vec{C}^{\vec{a}_3}_1 & \vec{C}^{\vec{a}_4}_1
 & & & & & & & & & & & & \\
\vec{C}^{\vec{a}_1}_2 & \mathbf{\vec{C}^{\vec{a}_2}_2} & \vec{C}^{\vec{a}_3}_2 & \vec{C}^{\vec{a}_4}_2
 & & & & & & & & & & & & \\
\vec{C}^{\vec{a}_1}_3 & \vec{C}^{\vec{a}_2}_3 & \mathbf{\vec{C}^{\vec{a}_3}_3} & \vec{C}^{\vec{a}_4}_3
 & & & & & & & & & & & & \\
\vec{C}^{\vec{a}_1}_4 & \vec{C}^{\vec{a}_2}_4 & \vec{C}^{\vec{a}_3}_4 & \mathbf{\vec{C}^{\vec{a}_4}_4}
 & & & & & & & & & & & & \\
 & & & & \mathbf{\vec{C}^{\vec{a}_5}_5} & \vec{C}^{\vec{a}_6}_5 & \vec{C}^{\vec{a}_7}_5 & 
  \vec{C}^{\vec{a}_8}_5 & & & & & & & & \\
 & & & & \vec{C}^{\vec{a}_5}_6 & \mathbf{\vec{C}^{\vec{a}_6}_6} & \vec{C}^{\vec{a}_7}_6 & 
 \vec{C}^{\vec{a}_8}_6 & & & & & & & & \\
 & & & & \vec{C}^{\vec{a}_5}_7 & \vec{C}^{\vec{a}_6}_7 & \mathbf{\vec{C}^{\vec{a}_7}_7} & 
 \vec{C}^{\vec{a}_8}_7 & & & & & & & & \\
 & & & & \vec{C}^{\vec{a}_5}_8 & \vec{C}^{\vec{a}_6}_8 & \vec{C}^{\vec{a}_7}_8 & 
 \mathbf{\vec{C}^{\vec{a}_8}_8} & & & & & & & & \\
 & & & & & & & & \mathbf{\vec{C}^{\vec{a}_9}_9} & \vec{C}^{\vec{a}_{10}}_9 & \vec{C}^{\vec{a}_{11}}_9 & 
 \vec{C}^{\vec{a}_{12}}_9 & & & & \\
 & & & & & & & & \vec{C}^{\vec{a}_9}_{10} & \mathbf{\vec{C}^{\vec{a}_{10}}_{10}} & 
 \vec{C}^{\vec{a}_{11}}_{10} & \vec{C}^{\vec{a}_{12}}_{10} & & & & \\
 & & & & & & & & \vec{C}^{\vec{a}_9}_{11} & \vec{C}^{\vec{a}_{10}}_{11} & 
 \mathbf{\vec{C}^{\vec{a}_{11}}_{11}} & \vec{C}^{\vec{a}_{12}}_{11} & & & & \\
 & & & & & & & & \vec{C}^{\vec{a}_9}_{12} & \vec{C}^{\vec{a}_{10}}_{12} & \vec{C}^{\vec{a}_{11}}_{12} & 
 \mathbf{\vec{C}^{\vec{a}_{12}}_{12}} & & & & \\
 & & & & & & & & & & & & \mathbf{\vec{C}^{\vec{a}_{13}}_{13}} & \vec{C}^{\vec{a}_{14}}_{13} & 
 \vec{C}^{\vec{a}_{15}}_{13} & \vec{C}^{\vec{a}_{16}}_{13} \\
 & & & & & & & & & & & & \vec{C}^{\vec{a}_{13}}_{14} & \mathbf{\vec{C}^{\vec{a}_{14}}_{14}} & 
 \vec{C}^{\vec{a}_{15}}_{14} & \vec{C}^{\vec{a}_{16}}_{14} \\
 & & & & & & & & & & & & \vec{C}^{\vec{a}_{13}}_{15} & \vec{C}^{\vec{a}_{14}}_{15} & 
 \mathbf{\vec{C}^{\vec{a}_{15}}_{15}} & \vec{C}^{\vec{a}_{16}}_{15} \\
 & & & & & & & & & & & & \vec{C}^{\vec{a}_{13}}_{16} & \vec{C}^{\vec{a}_{14}}_{16} & 
 \vec{C}^{\vec{a}_{15}}_{16} & \mathbf{\vec{C}^{\vec{a}_{16}}_{16}} \\
\end{pmatrix}
\end{equation}

\begin{equation}
E_0 = \Pi^4_{i=1} \vec{C}^{\vec{a}_{4i-3}}_{4i} \text{\hspace{8cm}} (\vec{C}^{\vec{a}_1}_4
\vec{C}^{\vec{a}_5}_8\vec{C}^{\vec{a}_9}_{12}\vec{C}^{\vec{a}_{13}}_{16})
\end{equation}
\begin{equation}
E_1 = \Pi^8_{i=1} \vec{C}^{\vec{a}_{4i-3}}_{4i-1} \text{\hspace{6cm}} (\vec{C}^{\vec{a}_1}_3
\vec{C}^{\vec{a}_2}_4\vec{C}^{\vec{a}_5}_{7}\vec{C}^{\vec{a}_6}_8\vec{C}^{\vec{a}_9}_{11}
\vec{C}^{\vec{a}_{10}}_{12}\vec{C}^{\vec{a}_{13}}_{15}\vec{C}^{\vec{a}_{14}}_{16})
\end{equation}
\begin{equation}
E_2 = \Pi^{12}_{i=1} \vec{C}^{\vec{a}_{4i-3}}_{4i-2} \text{\hspace{2cm}} (\vec{C}^{\vec{a}_1}_2
\vec{C}^{\vec{a}_2}_3\vec{C}^{\vec{a}_3}_4\vec{C}^{\vec{a}_4}_5\vec{C}^{\vec{a}_5}_6
\vec{C}^{\vec{a}_6}_7\vec{C}^{\vec{a}_7}_8\vec{C}^{\vec{a}_8}_9\vec{C}^{\vec{a}_9}_{10}
\vec{C}^{\vec{a}_{10}}_{11}\vec{C}^{\vec{a}_{11}}_{12}\vec{C}^{\vec{a}_{12}}_{13}
\vec{C}^{\vec{a}_{13}}_{14}\vec{C}^{\vec{a}_{14}}_{15}\vec{C}^{\vec{a}_{15}}_{16})
\end{equation}
\begin{equation}
E_3 = C = \Pi^{12}_{i=1} \vec{C}^{\vec{a}_{4i-3}}_{4i-3} \text{\hspace{3cm} (the main diagonal
on the matrix -- in bold font)}
\end{equation}
\begin{equation}
E_4 = \Pi^{12}_{i=1} \vec{C}^{\vec{a}_{4i-2}}_{4i-3} \text{\hspace{2cm}} (\vec{C}^{\vec{a}_2}_1
\vec{C}^{\vec{a}_3}_2\vec{C}^{\vec{a}_4}_3\vec{C}^{\vec{a}_5}_4\vec{C}^{\vec{a}_6}_5
\vec{C}^{\vec{a}_7}_6\vec{C}^{\vec{a}_8}_7\vec{C}^{\vec{a}_9}_8\vec{C}^{\vec{a}_{10}}_{9}
\vec{C}^{\vec{a}_{11}}_{10}\vec{C}^{\vec{a}_{12}}_{11}\vec{C}^{\vec{a}_{13}}_{12}
\vec{C}^{\vec{a}_{14}}_{13}\vec{C}^{\vec{a}_{15}}_{14}\vec{C}^{\vec{a}_{16}}_{15})
\end{equation}
\begin{equation}
E_5 = \Pi^8_{i=1} \vec{C}^{\vec{a}_{4i-1}}_{4i-3} \text{\hspace{6cm}} (\vec{C}^{\vec{a}_3}_1
\vec{C}^{\vec{a}_4}_2\vec{C}^{\vec{a}_7}_{5}\vec{C}^{\vec{a}_8}_6\vec{C}^{\vec{a}_{11}}_{9}
\vec{C}^{\vec{a}_{12}}_{10}\vec{C}^{\vec{a}_{15}}_{13}\vec{C}^{\vec{a}_{16}}_{14})
\end{equation}
\begin{equation}
E_6 = \Pi^4_{i=1} \vec{C}^{\vec{a}_{4i}}_{4i-3}  \text{\hspace{8cm}} (\vec{C}^{\vec{a}_4}_1
\vec{C}^{\vec{a}_8}_5\vec{C}^{\vec{a}_{12}}_{9}\vec{C}^{\vec{a}_{16}}_{13})
\end{equation}

%TODONE(3): write this section

\chapter{The product argument}

%TODO: not now; leave empty and write than in your e-mail.
% Acknowledgments

\appendix

%\chapter{Preliminaries}

\chapter{Definitions of cryptographic concepts}
\label{app:defs}

\begin{description}

\item \textbf{Shuffle} is a permutation.
%        [and reencryption (of homomorphic encyptions of ciphertexts) - does shuffling imply reencryption?
%        It doesn't seem so from the definition; check out shuffling in Wikipedia.
%        But in the context of the paper I don't know.
%       >> Communication operation: 
%      >> Computation operation: 
\item  \textbf{Mix-net} is a multi-party protocol which is used in e-voting or other 
applications which require anonymity.
It allows a group of senders to input a number of encrypted messages to the mix-net, which outputs 
them in random order. It is common to construct mix-nets from shuffles.
A common construction of mix-nets is to let the mix-servers take turns in shuffling the ciphertexts.
If the encryption scheme is semantically secure a shuffle of ciphertexts output by a mix-server does not
reveal the permutation of the messages. 
However, this means that a malicious mix-server could substitute
ciphertexts with encrypted votes for candidate X. 
So, we use an interactive argument to verify the 
correctness of a shuffle operation (soundeness), which reveals nothing about the permutation of 
the randomizers used (zero-knowledge).

\item \textbf{Zero-knowledge arguments} are used to verify the correctness of a shuffle operation 
        (since the shuffle is done upon a set of ciphertexts).
        
\item \textbf{Homomorphic encryption} is a form of encryption which allows specific types of 
computations to be performed on a ciphertext and produce
an encrypted result, which when decrypted matches the result of operations performed on the
plaintext.

\item \textbf{Cyclic group} is a group that is generated by a single element g. 
This element, g, is called the generator of the group. 
It consists of a set of elements with an invertible associative operation, and it
contains an element g such that any other element can be obtained from g by 
repeatedly applying the group operation or its inverse to g.

\item \textbf{Discrete logarithm}: let G be any group with its group operation denoted by multiplication. 
Let b and g be any elements of G. Then any integer k that solves the $b^k = g$ is termed a discrete 
logarithm of g to the base b.
      
\item \textbf{ElGamal encryption} is a type of partial homomorphic encryption;an 
asymmetric key encryption algorithm for public-key cryptography based on the Diffie-Hellman 
key exchange.
ElGamal encryption can be defined over any cyclic group G. 
Its security depends upon the difficulty of a specific problem in G related to computing 
discrete logarithms. 
ElGamal encruption consists of three components:
\begin{itemize}

\item the key generator
\item the encryption algorithm
\item the decryption algorithm
        
\end{itemize}

The security of ElGamal encryption depends upon the characteristics of the selected group G.

\item The \textbf{computational Diffie-Hellman assumption ({\sc cdh})} is the assumption that a certain 
computational problem within a cyclic group is hard. 
It is used to prove {\em confidentiality} of a cryptosystem.

\item When we talk about \textbf{confidentiality} of information, we are talking about 
protecting the information from disclosure to unauthorized parties.\footnote{
Taken from http://security.blogoverflow.com/2012/08/confidentiality-integrity-availability-the-three-components-of-the-cia-triad/}
A very key component of protecting information confidentiality would be encryption. 
Encryption ensures that only the right people (people who knows the key) can read 
the information. 
Encryption is very widespread in today's environment and can be found in almost 
every major protocol in use. 
A very prominent example will be {\sc ssl/tls}, a security protocol for communications 
over the internet that has been used in conjunction with a large number of internet 
protocols to ensure security.

\item The \textbf{decisional Diffie-Hellman assumption ({\sc ddh})} is a computational 
hardness assumption about a problem
involving discrete logarithms in cyclic groups. It is used to prove the security of many cryptographic
protocols, most notably the ElGamal and the Cramer-Shoup cryptosystems. 
The {\sc ddh} assumption states that given $g^a, g^b$ for uniformly and independently 
chosen a,b belonging to $Z_q$ the value $g^{ab}$ looks 
like a random element of G. {\sc ddh} is considered a stronger assumption than {\sc cdh}.
If {\sc cdh} holds in the underlying cyclic group G, the encryption function is one-way 
(easy to compute hard to invert).
If {\sc ddh} holds in G, ElGamal achieves {\em semantic security}.
To achieve chosen ciphertext attack security, the scheme has to be further modified or an appropriate 
padding scheme must be used.

\item In cryptography, a cryptosystem is \textbf{semantically secure} if any probabilistic, 
polynomial-time algorithm ({\sc ppta}) that is given the ciphertext of a certain message m 
(taken from any distribution of messages), and the message's length, cannot determine 
any partial information on the message with probability non-negligibly higher than all other 
{\sc ppta}'s that only have access to the message length (and not the ciphertext).\footnote{
Taken from http://en.wikipedia.org/wiki/Semantic\_security}
In other words, knowledge of the ciphertext (and length) of some unknown message 
does not reveal any additional information on the message that can be feasibly extracted. 
This concept is the computational complexity analogue to Shannon's concept of perfect secrecy. 
Perfect secrecy means that the ciphertext reveals no information at all about the plaintext, 
whereas semantic security implies that any information revealed cannot be feasibly extracted.

\item \textbf{Confidentiality, semantic security, and perfect secrecy}.\footnote{
Taken from http://crypto.stackexchange.com/questions/14321/perfect-secrecy-one-time-semantic-security-secure-prg}
Note first that an adversary has some a priori knowledge of the message. 
We can capture that by e.g. having the adversary choose two messages and 
then flipping a fair coin to decide which one to encrypt.
Next, note that the adversary wants to learn something from the ciphertext, 
his a posteriori knowledge of the message. 
We capture that by having the adversary guess which message was encrypted.
If the adversary cannot guess correctly which message was encrypted with 
probability significantly different from 1/2, then we have confidentiality.
That was the definition of confidentiality.
(More complex definitions are possible using a probability space and a function 
from the message space to 0 and 1.)
Perfect security is about confidentiality against arbitrary adversaries. 
Now we can prove many theorems about perfect security, and one of them talks 
about the number of keys taking a given message to a given ciphertext.
Semantic security is about confidentiality against computationally bounded adversaries. 
Unlike for perfect security, there aren't so many theorems we can prove about semantic security, 
which means that we know less about what is required to achieve semantic security.
So the difference between perfect and semantic security is just about which adversaries we 
consider, nothing else.

\item \textbf{Initialization vector} is a fixed-size input to a cryptographic primitive that is 
required to be random or pseudo-random. 
Randomization is crucial for encryption schemes to achieve semantic security that is 
that repeated usage of the scheme under the same key does not allow an attacker to infer relationships 
between segments of the encrypted message. 
To hide patterns in encrypted data and to avoid using a new key
for each block a method is needed to randomize the input data. 
For this purpose, ciphertext from one block
gets intermixed with ciphertext from another block. 
To randomize the first block additional input is needed.
This is where the initialization vector comes in handy. 
For instance the Cipher-Block Chaining mode adds the
initialization vector to the first plaintext block, the first produced ciphertext block to the second 
plaintext block, and so on. The ultimate goal is semantic security. 
For more see initialization vector:motivation in Wikipedia.

\item A \textbf{commitment scheme} allows one to commit to a chosen value, 
while keeping it hidden to others, with the ability to reveal the committed value later. 
Commitment schemes are designed so that a party cannot change
the value of statement after they have committed to it; that is commitment schemes are 
binding. 
You can see the rock-paper-scissors example in Wikipedia under commitment scheme entry.

\end{description}


% Bibliography
\bibliographystyle{plain}
\bibliography{minimal-shuffle-doc}


\end{document}
